\chapter{Indledning}

Dette dokument vil beskrive udviklingsmiljøet brugt til udviklig af Converge-app. Dette er inddelt til de forskellige roller: Udvikler, Administrator og Supporter.

Dokumentet vil beskrive hvordan det mindst mulige udviklingsmiljø kan blive sat op.

\section{Udvikler}

Udvikleren har primært brug for et sted at skrive kode, indlevere det samt at få det til at køre i produktion.

Derfor har udvikleren brug for værktøjer til dette formål.

\subsubsection{IDE}

IDE står for integrated development environment, og skal komme med alt det en udvikler har brug for at skrive det kode ønsket. Til converge er der dog mest brugt en Text Editor i form af Visual Studio Code (VSCode). VSCode er en editor af Microsoft, der tager inspiration fra mange andre tidligere editors, såsom Sublime, Atom og Microsofts egen Visual Studio.

VSCode er cross-platform og er nemt at sætte op. For at skrive kode til Converge skal der dog bruges nogle extensions til VSCode.

\begin{itemize}
    \item ES7 React/Redux
    \item C#
    \item Kubernetes
    \item Docker
    \item Git
    \item GitLens
    \item LaTeX Workshop
    \item PlantUML
\end{itemize}

Det er dog ikke alle værktøjer som alle udviklere får brug for, f.eks. Hvis en udvikler kun skal arbejde med Converge-spa, ville det være nok at have ES7 React/Redux og Git. Men til en Fullstack udvikler, kræver det alle værktøjer.

\subsubsection{CLI værktøjer}

Til at administrere Udrulningen af appen er der brug for værktøjer til henholdsvis Docker, Google og Kubernetes.

Dette er i form af:

\begin{itemize}
    \item Docker (#docker)
    \item Google SDK (#gcloud)
    \item Kubernetes (#kubectl)
    \item Helm (#helm)
    \item Dotnet SDK
    \item git
    \item NodeJS
    \item Optional
    \begin{itemize}
        \item Minikube
    \end{itemize}
\end{itemize}

Docker bruges til at teste pakningen af applikationer. Google SDK til at vedligeholde alle de hostede google resourcer, (Cloud DNS, Google Kubernetes Engine osv.) Kontakt converge.kjuulh@gmail.com for adgang til resourcer. Kubernetes til at debugge og styre udrulningen af applikationer på produktions cluster, og til Udvikling på Minikube hvis downloaded). Derefter kan helm downloades, dette er ikke et krav, men er rart for at kunne styre udrulningen af applikationer, samt bedre værktøjer til at lave nye charts til kubernetes. Dotnet SDK bruges til at lave nye projekter, men mere for at kunne køre server kode. Dette er et krav, hvis altså Visual Studio 2017/2019 ikke er downloaded. Git, git er et krav, da git er påkrævet for at samarbejde om koden, og til at kunne levere kode til produktion. NodeJS er brugt til udviklingen af Converge-spa og er påkrævet for at kunne køre hjemmesiden i udviklingsmiljøet. Minikube kan downloades og køres, dette kræver at man har en del andre applikationer installere, men giver muligheden for at have et lokalt kubernetes cluster.

\subsubsection{Tracing}

Sporing (Tracing) i produktion er et af de udskrevne krav for at køre en decentral microservices arkitektur. Sporing giver muligheden for at binde applikationer sammen, som ikke køre i samme process, samt at kunne se hvad de individuelle brugere gør, og hvordan og hvornår fejl opstår. Tracing giver et overblik over applikationen, som man normalt ellers ikke ville have.

Tracing er givet ved en kobling af de forskellige server applikationer, kørende dotnet samt Jaeger (OpenTracing standard). Jaeger gemmer sine spor (Traces/Frames) i Elastic search og tilbyder et web interface hvor de forskellige koblinger og kald kan blive hentet.

\fxfatal{Jaeger link}

\subsubsection{Logning}

Logning er lige så vigtigt som Sporing, men meget mere granulært i dets brug. Man kunne sige at logning kunne det samme som tracing, men logning er sværre at bruge, men er også beregnet til andre formål. Logning er brugt til at holde styr på forskellige statistikker såsom fejl pr time, antal brugere pr dag, antal api kald pr applikation, de mest belastede applikationer osv. 

Logning er givet ved en kobling af de forskellige server applikationer, kørende dotnet, samt ELK stack (Elastic search, logstash og Kibana), hvor EK er brugt mest. Dotnet interagere med Elastic search, og Kibana tilbyder et interface til Elastic search. Med Kibana kan queries samles og grafer kan defineres, så Ops personale kan holde overblik over systemets helbred.

\fxfatal{ELK link}

\subsubsection{Monitoring}

Monitoring er vigtigt for at holde overblik over hvor mange og hvor godt helbred ens appliktioner har det. Kubernetes er et selv helene system, men det er svært at holde styr på hvilke applikationer der køre hvor, og hvor godt helbred de har. Prometheus er brugt til dette. Prometheus kan skrabe forskellige endpoints udstedt af Kubernetes selv, og de applikationer kørende i kubernetes. Prometheus kan derefter levere dette videre til et utal af interfaces. Indkluderende dets eget.

\fxfatal{Prometheus link}

\subsubsection{GitHub \& Feature Branching}

Til at hoste Converge kode er GitHub brugt, både som Repository manager, og til administrering af feature requests og bugs. Til alt kode er feature branching brugt. Ved feature branching laves en branch til noget kode, derefter er det kode reviewet i et pull request og merged ind i master, hvorefter det er udrullet i produktion. Dette betyder der bliver kørt unit tests, systemtests og accepttest, både før og efter applikationen er udrullet til produktion. Feature branching og GitHub gør det muligt at kræve et kode review, så der kommer flere øjne på det kode der eventuelt skal eller ikke skal i produktion. Dette er også kaldt Continuous Integration \fxfatal{CI}, da kode kontinuert blive updateret fra produktion mod produktion.